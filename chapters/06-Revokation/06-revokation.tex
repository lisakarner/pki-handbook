\chapter{Certificate Revocation}

\section{CRL – Certificate Revocation Lists}

\subsection{Erstellung und Veröffentlichung einer CRL}
\subsubsection*{Ausgangssituation und Zielsetzung}
In diesem Labor wird eine Certificate Revocation List (CRL) erstellt und über einen Webserver (Nginx) bereitgestellt. Die vorhandene CA-Struktur wurde bereits zuvor in XCA aufgebaut. Ziel ist es, ein Serverzertifikat mit einem CRL Distribution Point (CDP) zu versehen, dieses anschließend zu widerrufen und die zugehörige CRL über Nginx zugreifbar zu machen. Am Ende soll der Browser beim Inspizieren des Zertifikats korrekt auf den veröffentlichten CRL-Pfad verweisen.

\subsubsection*{Anpassung des Serverzertifikats in XCA}

Öffnen der bestehenden CA-Struktur in XCA.

\begin{figure}[H]
    \centering
    \includegraphics[width=0.8\textwidth]{xcacrl1.png}
    \caption*{XCA Übersicht}
\end{figure}

Auswahl des entsprechenden Serverzertifikats, das angepasst werden soll.

Im Reiter \texttt{Extensions} bzw. \texttt{Extensions → CRL Distribution Points} den Distribution Point eintragen:

\begin{verbatim}
http://192.168.1.1/crl/cacrl.pem
\end{verbatim}

\begin{figure}[H]
    \centering
    \includegraphics[width=0.8\textwidth]{xcacrl2.png}
    \caption*{CRL Distribution Point}
\end{figure}

\subsubsection*{Widerruf des Serverzertifikats}

Das soeben angepasste Serverzertifikat wird in XCA ausgewählt.

Über \texttt{Revocation / Revoke} wird das Zertifikat widerrufen.

\begin{figure}[H]
    \centering
    \includegraphics[width=0.8\textwidth]{xcacrl3.png}
    \caption*{Widerruf in XCA}
\end{figure}

Dadurch wird klar, dass dieses Zertifikat nicht mehr gültig ist und künftig in der CRL eingetragen werden muss.

\subsubsection*{Erstellen der CRL des Intermediate-Zertifikats}

Da das widerrufene Serverzertifikat vom Intermediate-Zertifikat signiert wurde, wird die CRL auf Ebene dieser Intermediate-CA erzeugt:

\begin{itemize}
    \item Auswahl der Intermediate-CA in XCA.
    \item Im Bereich \texttt{CRLs} die Option \texttt{New CRL} wählen.
    \item CRL-Parameter wie Gültigkeitsdauer übernehmen oder anpassen.
    \item CRL erzeugen und als Datei exportieren (PEM-Format).
\end{itemize}

Ausgabeformat:

\begin{verbatim}
cacrl.pem
\end{verbatim}

\begin{figure}[H]
    \centering
    \includegraphics[width=0.8\textwidth]{xcacrl4.png}
    \caption*{CRL Erstellung}
\end{figure}

\subsubsection*{Bereitstellen der CRL über Nginx}

\paragraph*{Ablage der CRL auf dem Webserver}
Auf der Ubuntu-VM den vorgesehenen Ordner unter \texttt{/var/www/html/} anlegen (falls nicht vorhanden):

\begin{verbatim}
sudo mkdir -p /var/www/html/crl
\end{verbatim}

Die zuvor exportierte Datei \texttt{cacrl.pem} nach diesem Verzeichnis kopieren:

\begin{verbatim}
sudo cp /pfad/zur/cacrl.pem /var/www/html/crl/
\end{verbatim}

\begin{figure}[h!]
    \centering
    \caption*{Dateistruktur auf der VM}
\end{figure}

\paragraph*{Nginx neu starten}
Nach dem Platzieren der Datei muss Nginx neu geladen werden, damit die CRL korrekt ausgeliefert wird:

\begin{verbatim}
sudo systemctl restart nginx
\end{verbatim}

\begin{figure}[h!]
    \centering
    \caption*{Bestätigung des Neustarts}
\end{figure}

\subsubsection*{Überprüfung im Browser}

\begin{itemize}
    \item Zugriff auf den Webdienst mit dem betroffenen Serverzertifikat herstellen.
    \item Im Browser das Zertifikat inspizieren (Zertifikatsdetails öffnen).
    \item Unter \texttt{CRL Distribution Points} sollte nun der gesetzte Link erscheinen:
\end{itemize}

\begin{verbatim}
http://192.168.1.1/crl/cacrl.pem
\end{verbatim}

Test: Der Link lässt sich im Browser öffnen – die CRL kann theoretisch heruntergeladen werden.

\subsubsection*{Ergebnis}

Durch die korrekte Konfiguration des CRL Distribution Points, das Widerrufen des Serverzertifikats und das Erstellen sowie Bereitstellen der CRL über Nginx kann der Browser die CRL erfolgreich finden und anzeigen. Damit ist der Widerrufsmechanismus funktional implementiert.


\subsection{Verhalten moderner Browser bei CRLs}

\subsubsection*{Einordnung}
Nachdem im vorherigen Abschnitt gezeigt wurde, wie eine CRL mit XCA erzeugt und über einen Webserver wie Nginx bereitgestellt wird, stellt sich in der Praxis die Frage, welche Clients diese Informationen tatsächlich auswerten. Besonders relevant ist dabei das Verhalten gängiger Browser, da diese bestimmen, ob ein Widerruf im Alltag überhaupt erkannt wird. Im Kontext unseres Setups – einer eigenen CA mit veröffentlichtem CRL Distribution Point – ist insbesondere Mozilla Firefox bemerkenswert, da der Browser die bereitgestellte CRL trotz korrekter Konfiguration in vielen Fällen nicht abruft.

\subsubsection{Mozilla Firefox}
\paragraph*{Einleitung}

In einer Public-Key-Infrastruktur (PKI) dienen Certificate Revocation Lists (CRLs) und OCSP-Abfragen eigentlich dazu, festzustellen, ob ein Zertifikat widerrufen wurde, bevor seine reguläre Gültigkeit abläuft. Die dazu nötigen Informationen – wie der CRL Distribution Point (CDP) oder die Authority Information Access (AIA) – liegen im X.509-Zertifikat eingebettet vor.

In modernen Browsern wird dieses Verhalten jedoch unterschiedlich interpretiert. Gerade im Kontext unserer selbst betriebenen CA-Struktur ist wichtig zu verstehen, dass Firefox diese Felder nicht zuverlässig auswertet, selbst wenn CRLs wie zuvor eingerichtet korrekt über HTTP erreichbar sind.

\paragraph*{Verhalten von Firefox}

Firefox setzt zur Zertifikatsvalidierung auf die \textit{Network Security Services (NSS)}-Bibliothek. Laut einer Diskussion in der offiziellen Mozilla Developer Mailingliste ruft Firefox CRLs grundsätzlich nicht automatisch anhand der im Zertifikat hinterlegten CDPs ab – mit Ausnahme bestimmter Extended-Validation-Zertifikate:

\begin{quote}
    ``Firefox uses NSS’s feature to fetch CRLs~\ldots{} but only for EV certificate chains.'' \\
    (Mozilla.dev.security.policy, 2010)
\end{quote}

Quelle: \url{https://groups.google.com/g/mozilla.dev.security.policy/c/piOzRNNcgy0}

Auch andere Analysen bestätigen, dass CRLs von Firefox im Normalfall vollständig ignoriert werden:

\begin{quote}
    ``Firefox allows you to check for revoked certificates via the OCSP method, but it doesn’t use the CRL at all.'' \\
    (F5 Networks, 2016)
\end{quote}

Quelle: \url{https://community.f5.com/kb/technicalarticles/security-sidebar-my-browser-has-no-idea-your-certificate-was-just-revoked/281100}

Anstelle klassischer CRL- oder OCSP-Abfragen setzt Mozilla zunehmend auf ein alternatives Widerrufssystem namens \textit{CRLite}, das Widerrufsdaten in komprimierter Form zentral an Firefox-Nutzer verteilt:

\begin{quote}
    ``CRLite pushes bulk certificate revocation information to Firefox users, reducing the need to actively query such information one by one.'' \\
    (Mozilla Security Blog, 2020)
\end{quote}

Quelle: \url{https://blog.mozilla.org/security/2020/01/21/crlite-part-3-speeding-up-secure-browsing}

Weiterhin wurde angekündigt, dass mit Firefox 142 das Online Certificate Status Protocol (OCSP) für DV-Zertifikate vollständig deaktiviert wird:

\begin{quote}
    ``We will be disabling OCSP for domain validated certificates in Firefox 142.'' \\
    (Mozilla Hacks, 2025)
\end{quote}

Quelle: \url{https://hacks.mozilla.org/2025/08/crlite-fast-private-and-comprehensive-certificate-revocation-checking-in-firefox}


\paragraph*{Quellen}

\begin{itemize}
    \item Mozilla.dev.security.policy (2010). Discussion: Firefox CRL behavior. \\
    \url{https://groups.google.com/g/mozilla.dev.security.policy/c/piOzRNNcgy0}

    \item F5 Networks (2016). \textit{My browser has no idea your certificate was just revoked.} \\
    \url{https://community.f5.com/kb/technicalarticles/security-sidebar-my-browser-has-no-idea-your-certificate-was-just-revoked/281100}

    \item Mozilla Security Blog (2020). \textit{CRLite: Speeding Up Secure Browsing.} \\
    \url{https://blog.mozilla.org/security/2020/01/21/crlite-part-3-speeding-up-secure-browsing}

    \item Mozilla Hacks (2025). \textit{CRLite: Fast, Private, and Comprehensive Certificate Revocation Checking in Firefox.} \\
    \url{https://hacks.mozilla.org/2025/08/crlite-fast-private-and-comprehensive-certificate-revocation-checking-in-firefox}
\end{itemize}


\subsubsection{Google Chrome}
\section{Chrome}
\subsection{OCSP}

OCSP wurde bereits 1999 standardisiert, hatte jedoch für Browser lange keine hohe Priorität. 
Da die meisten Browser OCSP kaum implementierten, investierten Zertifizierungsstellen nur 
wenig in die notwendige Infrastruktur. Sobald OCSP-Server stärker unterstützt wurden, 
zeigten sich deutliche Performanceprobleme, da jede Abfrage zusätzliche Latenz verursachte 
und von der Verfügbarkeit des OCSP-Servers abhing (Quelle: FeistyDuck, 
\url{https://www.feistyduck.com/newsletter/issue_121_the_slow_death_of_ocsp}).

\bigskip   

In der Folge wurde Soft-Fail-OCSP zur Norm. Beim Soft-Fail wird ein Zertifikat selbst dann 
akzeptiert, wenn die OCSP-Anfrage fehlschlägt oder nicht verfügbar ist (Quelle: Radiusaas FAQ, 
\url{https://docs.radiusaas.com/other/faqs/ocsp-soft-fail-consequences}).

\bigskip   

In Unternehmensumgebungen kann OCSP zwar aktiviert oder erzwungen werden, praktisch handelt es 
sich jedoch meist um Soft-Fail-OCSP, da striktes Hard-Fail-Verhalten (Blockieren bei Ausfall 
des OCSP-Servers) zu hoher Fehleranfälligkeit führt (Quelle: Chromium Security, 
\url{https://www.chromium.org/Home/chromium-security/crlsets/}).

\bigskip   

Google Chrome unterstützt keine Live-OCSP-Abfragen, sondern verwendet stattdessen 
CRLSets und OCSP-Stapling (Quelle: SSL.com, 
\url{https://www.ssl.com/de/Artikel/OCSP-Stapling-%E2%80%93-sichere-und-effiziente-Zertifikatsvalidierung/}).



Chrome prüft Zertifikatswiderrufe außerdem nicht über klassische OCSP-Abfragen, sondern 
setzt auf alternative Mechanismen (Quelle: Uwe Gradenegger, 
\url{https://www.gradenegger.eu/de/google-chrome-prueft-sperrstatus-von-zertifikaten-nicht/}).

\bigskip   

\begin{itemize}
	\item \textbf{CRLSets} dienen als Notfallmechanismus für größere Sicherheitsvorfälle 
	und werden von Google gepflegt und über Browser-Updates verteilt.
	
	\item \textbf{OCSP-Stapling} wird unterstützt: Der Webserver fragt selbstständig den 
	OCSP-Responder ab, speichert die signierte OCSP-Response im Cache und sendet diese 
	beim TLS-Handshake „gestapelt“ an den Browser (Quelle: SSL.com, 
	\url{https://www.ssl.com/de/Artikel/OCSP-Stapling-%E2%80%93-sichere-und-effiziente-Zertifikatsvalidierung/}).
\end{itemize}

\bigskip   

\textbf{Fazit:} OCSP konnte sich trotz früher Standardisierung nicht als zuverlässige 
Sperrprüfung etablieren, da Latenz und Serververfügbarkeit zu Performanceproblemen führten. 
Das Soft-Fail-Prinzip reduzierte den Sicherheitsnutzen erheblich. Google Chrome verzichtet 
daher vollständig auf Live-OCSP und setzt stattdessen auf CRLSets und OCSP-Stapling. 
Unternehmensumgebungen nutzen zwar teilweise OCSP, jedoch fast ausschließlich in der 
Soft-Fail-Variante, um Verbindungsprobleme zu vermeiden.


\subsubsection{Microsoft Edge}
\input{chapters/06-Revokation/CRL_im_Browser/Edge.tex}

\subsubsection{Apple Safari}
\input{chapters/06-Revokation/CRL_im_Browser/Safari.tex}

\subsubsection*{Allgemeine Herausforderungen und Problematik bei Zertifikatswiderrufen}
Bevor die verbleibenden technischen Einschränkungen erläutert werden, ist wichtig zu verstehen, dass die Unterstützung für klassische Widerrufsmechanismen wie CRLs und OCSP in modernen Browsern stark variiert.
Während einige Browser diese Verfahren gar nicht mehr nutzen, unterstützen andere sie nur rudimentär oder mit deutlichen Einschränkungen.

Dies führt dazu, dass selbst korrekt veröffentlichte CRLs – wie in unserem Setup – häufig nicht oder nur teilweise von Clients ausgewertet werden.
Die Folge ist, dass widerrufene Zertifikate in der Praxis nicht zuverlässig erkannt werden.
Dies betrifft insbesondere interne PKI-Umgebungen und private CAs, die nicht in globale Widerrufssysteme wie CRLite integriert sind.

\paragraph*{Vertrauenskette verliert ihre Dynamik}
Eine klassische PKI setzt darauf, dass jeder Browser regelmäßig prüft, ob ein Glied der Chain of Trust widerrufen wurde.
Wird dieser Schritt übersprungen oder nicht unterstützt, verlieren Widerrufe ihre Wirkung.
Ein kompromittiertes Zertifikat bleibt dann faktisch bis zum Ablaufdatum gültig – unabhängig davon, ob es offiziell widerrufen wurde.

\paragraph*{Private CAs sind vom CRLite-System ausgeschlossen}
CRLite deckt ausschließlich Widerrufsdaten von öffentlichen CAs aus dem Mozilla-Root-Store ab.
Private oder interne CAs – wie in unserem Labor – sind in diesem System nicht enthalten.
Dadurch werden Widerrufe solcher Zertifikate von Browsern wie Firefox weder berücksichtigt noch automatisch geprüft.

\paragraph*{Fazit}
Auch wenn wir in unserem Setup eine CRL erfolgreich erzeugt und korrekt veröffentlicht haben, bedeutet dies nicht, dass alle Browser diese Informationen nutzen.
Gerade Firefox ignoriert CRL Distribution Points vollständig und verlässt sich stattdessen auf alternative Mechanismen wie OCSP (nur eingeschränkt) oder CRLite.

Damit wird deutlich: Der technische Widerruf eines Zertifikats ist das eine – die tatsächliche Auswertung durch Clients jedoch eine völlig andere Herausforderung, die in modernen Browsern zunehmend unzuverlässig oder gar nicht mehr stattfindet.



% ==========================================================
%           **NEUER HAUPTPUNKT: OCSP**
% ==========================================================
\newpage

\section{OCSP – Online Certificate Status Protocol}


