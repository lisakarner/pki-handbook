Bevor die verbleibenden technischen Einschränkungen erläutert werden, ist wichtig zu verstehen, dass die Unterstützung für klassische Widerrufsmechanismen wie CRLs und OCSP in modernen Browsern stark variiert.
Während einige Browser diese Verfahren gar nicht mehr nutzen, unterstützen andere sie nur rudimentär oder mit deutlichen Einschränkungen.

Dies führt dazu, dass selbst korrekt veröffentlichte CRLs – wie in unserem Setup – häufig nicht oder nur teilweise von Clients ausgewertet werden.
Die Folge ist, dass widerrufene Zertifikate in der Praxis nicht zuverlässig erkannt werden.
Dies betrifft insbesondere interne PKI-Umgebungen und private CAs, die nicht in globale Widerrufssysteme wie CRLite integriert sind.

\paragraph*{Vertrauenskette verliert ihre Dynamik}
Eine klassische PKI setzt darauf, dass jeder Browser regelmäßig prüft, ob ein Glied der Chain of Trust widerrufen wurde.
Wird dieser Schritt übersprungen oder nicht unterstützt, verlieren Widerrufe ihre Wirkung.
Ein kompromittiertes Zertifikat bleibt dann faktisch bis zum Ablaufdatum gültig – unabhängig davon, ob es offiziell widerrufen wurde.

\paragraph*{Private CAs sind vom CRLite-System ausgeschlossen}
CRLite deckt ausschließlich Widerrufsdaten von öffentlichen CAs aus dem Mozilla-Root-Store ab.
Private oder interne CAs – wie in unserem Labor – sind in diesem System nicht enthalten.
Dadurch werden Widerrufe solcher Zertifikate von Browsern wie Firefox weder berücksichtigt noch automatisch geprüft.

\paragraph*{Fazit}
Auch wenn wir in unserem Setup eine CRL erfolgreich erzeugt und korrekt veröffentlicht haben, bedeutet dies nicht, dass alle Browser diese Informationen nutzen.
Gerade Firefox ignoriert CRL Distribution Points vollständig und verlässt sich stattdessen auf alternative Mechanismen wie OCSP (nur eingeschränkt) oder CRLite.

Damit wird deutlich: Der technische Widerruf eines Zertifikats ist das eine – die tatsächliche Auswertung durch Clients jedoch eine völlig andere Herausforderung, die in modernen Browsern zunehmend unzuverlässig oder gar nicht mehr stattfindet.
